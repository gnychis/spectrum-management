% !TEX root = main.tex

\abstract{
Heterogeneity has become an increasing problem in the wireless spectrum, breaking down spectrum sharing, exacerbating interference, and making highly inefficient use of already scarce spectrum.  Coexistence techniques can alleviate such interference, however they are difficult to deploy (requiring changes across many layers), often incurring overhead, and they are typically short-lived due to rapid changes in technologies and standards.  While spectrum management can be a more efficient and long-term solution, current models have remained relatively homogeneous and Wi-Fi centric.

In this paper, we present a novel heterogeneous spectrum assignment model to properly organize current environments and reduce interference within them.  This model incorporates differences in the physical and MAC layers of radios in the environment, as well as their spatial dynamics (i.e., what interferes with what).  We represent these properties using an annotated hypergraph, and introduced a non-linear mixed integer program to find efficient heterogeneous spectrum organizations.  Results from a heterogeneous testbed show the spectrum organizations to be highly efficient, as well as able to account for various PHY and MAC conflicts.

%To represent the environment, we introduce a hypergraph-based model.  Then, leveraging the hypergraph, we 

% To do so, we first introduce a hypergraph-based model to accurately represent a heterogeneous environment and properties of it.  Then, we introduce a metric to estimate the performance of a network

%Our model is generic, yet descriptive.  By maintaining generic

%Proper heterogeneous spectrum management can be a more efficient and long-term solution
%
% techniques can alleviate such interference, however these approaches are difficult to deploy and
%
%In this paper, we present a heterogeneous spectrum assignment model and algorithm towards a long term solution to cross-technology interference.  }
