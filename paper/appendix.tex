%% !TEX root = main.tex
%
%\textbf{Deriving Frequency Overlap:}  To derive overlap, we begin with calculating the low and high frequencies of a radio $R_i$'s operational band.  For a radio $R_i$ with a center frequency $F_{ik}$ and bandwidth $B_{ij}$, the low frequency of its operational band would be $L_{ik} = F_{ik} - (B_{ij}/2)$, and its high frequency of its band: $H_{ik} = F_{ik} + (B_{ij}/2)$.  This frequency band would overlap with another band belonging to network $N_r$ with a center frequency of $F_{rj}$ and bandwidth $B_i$ under the following conditions:
%
%\vspace{-0.1in}
%\begin{equation}
%\begin{split}
%O_{i_kr_j} \leftarrow 
%\begin{Bmatrix}
%1:& L_{ik} \in [L_{rj}, H_{rj}]~~or~~L_{rj} \in [L_{ik}, H_{ik}] \\
%0:&  L_{ik} \notin [L_{rj}, H_{rj}]~and~L_{rj} \notin [L_{ik}, H_{ik}] 
%\end{Bmatrix}
%\end{split}
%\label{overlap}
%\end{equation}